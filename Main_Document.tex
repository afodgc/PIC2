% #############################################################################
% This is the MAIN DOCUMENT of the Thesis MSc TEMPLATE.
% The content for the Thesis MSc is to be written in separate documents
% located in the folder ./Chapters
%         Aknowledgments.tex
%         Abstract.tex
%         KeyWords.tex
%         Resumo.tex
%         PalavrasChave.tex
%         Acronyms.tex
%         Front_Cover.tex
%         Chapter_1.tex ....Chapter_2 .....
%         ApendixA.tex ... ApendixB.tex...
% -----------------------------------------------------------------------------
% The class "istulthesis" is based on the standard LaTeX 'report' class.
% It can be used for Instituto Superior Tecnico thesis, as it follows the
% https://www.overleaf.com/6139349192mwrcjcczyjfw
% regulations published by the Scientific Council of IST.
% The class defines the document style.
% IST requires the thesis to be written in Arial or similar.
% Two arguments in '\documentclass' allow you to define the thesis font:
% 'Helvetica' and 'AvantGarde', which transforms
% the default LaTeX font into Helvetica or AvantGarde, respectively.
% #############################################################################
% The document is automatically set for english or portuguese by just selecting
% the MAIN LANGUAGE in file 'Thesis-MSc-Preamble_commands.tex'
% #############################################################################
% Thesis-MSc
% Version 2.0, August 2018
% BY: Rui Santos Cruz, rui.s.cruz@tecnico.ulisboa.pt
% #############################################################################
% !TEX root = ./main.tex
% -----------------------------------------------------------------------------
%
\documentclass[defaultstyle,10pt,Helvetica]{istulthesis}
%
% -----------------------------------------------------------------------------
% The Preamble document contains all the necessary Packages for typesetting
% Modify it to suit your needs
% -----------------------------------------------------------------------------
% #############################################################################
% Preamble for Thesis-MSc in English or Portuguese
% Required Packages and commands
% --> Please Choose the MAIN LANGUAGE for the Thesis in package BABEL (below)
% !TEX root = ./main.tex
% #############################################################################
% Thesis-MSc
% Version 2.0, August 2018
% BY: Rui Santos Cruz, rui.s.cruz@tecnico.ulisboa.pt
% #############################################################################
%
% -----------------------------------------------------------------------------
% PACKAGES ucs, utf8x, babel, iflang:
% -----------------------------------------------------------------------------
% The 'ucs' package provides support for using UTF-8 in LaTeX documents.
% However in most situations it is not required.
\usepackage{ucs}
% The 'utf8x' package contains support for using UTF-8 as input encoding.
\usepackage[utf8x]{inputenc}
% The 'babel' package may correct some hyphenation issues of LaTeX.
% Select your MAIN LANGUAGE for the Thesis with the 'main=' option.
\usepackage[main=english,portuguese]{babel}
% The 'iflang' package is used to help determine the language being used.
\usepackage{iflang}

% -----------------------------------------------------------------------------
% PACKAGE scrbase:
% -----------------------------------------------------------------------------
% The 'scrbase' package is used to help redefining document structure.
\usepackage{scrbase}
% -----------------------------------------------------------------------------
% PACKAGE mathtools, amsmath, amsthm, amssymb, amsfonts, nicefrac:
% -----------------------------------------------------------------------------
% These packages are typically required.
% Among many other things they add the possibility to put symbols in bold
% by using \boldsymbol (not \mathbf); defines additional fonts and symbols;
% adds the \eqref command for citing equations.
\usepackage{mathtools, amsmath, amsthm, amssymb, amsfonts}
\usepackage{nicefrac}
%
% -----------------------------------------------------------------------------
% PACKAGE tikz:
% -----------------------------------------------------------------------------
% Tikz  for creating graphics programmatically.
\usepackage{tikz}
\usetikzlibrary{shapes.geometric, arrows, positioning}
% -----------------------------------------------------------------------------
% PACKAGES array, booktabs, multirow, colortbl, ctable, spreadtab:
% -----------------------------------------------------------------------------
% These packages are most usefull for advanced tables.
% 'multirow' allows to join rows throuhg the command \multirow which works
% similarly with the command \multicolumn.
% The 'colortbl' package allows to color the table (foreground and background)
% The 'ctable' package provides commands to easily typeset centered or left or
% right aligned tables.
% The package 'booktabs' provide some additional commands to enhance
% the quality of tables
% The 'longtable' package is only required when tables extend beyond the length
% of one page, which typically does not happen and should be avoided
\usepackage{array}
\usepackage{booktabs}
\usepackage{multirow}
\usepackage{colortbl}
\usepackage{ctable}
\usepackage{spreadtab}
\usepackage{longtable}


\usepackage{outlines}

%\usepackage{minipage}
%
% -----------------------------------------------------------------------------
% PACKAGES graphicx, subfigure:
% -----------------------------------------------------------------------------
% The package 'graphicx' supports formats PNG and JPG.
% Package 'subfigure' allows to place figures within figures with own caption.
% For each of the subfigures use the command \subfigure.

\usepackage{subcaption}

\usepackage{graphicx}
%\usepackage[hang,small,bf,tight]{subfigure}

%\usepackage{subfig}






%
% -----------------------------------------------------------------------------
% PACKAGE caption:
% -----------------------------------------------------------------------------
% The 'caption' package offers customization of captions in floating
% environments such figure and table
% \usepackage[hang,small,bf]{caption}
\usepackage[format=hang,labelfont=bf,font=small]{caption}
% the following customization adds vertical space between caption and the table
\captionsetup[table]{skip=10pt}
%
% -----------------------------------------------------------------------------
% PACKAGE algorithmic, algorithm, algorithm2e:
% -----------------------------------------------------------------------------
% These packages are required if you need to describe an algorithm.
% The preference is for using 'algorithm2e'
%\usepackage{algorithmic}
%\usepackage[chapter]{algorithm}



%%% HERE HEREE HERE HERE
%%% HERE HEREE HERE HERE
%%% HERE HEREE HERE HERE%%% HERE HEREE HERE HERE
%%% HERE HEREE HERE HERE
%%% HERE HEREE HERE HERE
%%% HERE HEREE HERE HERE%%% HERE HEREE HERE HERE
%\usepackage[ruled,vlined,algochapter,norelsize,\languagename]{algorithm2e}

%
%\usepackage{subfigure}

% -----------------------------------------------------------------------------
% PACKAGE listings
% -----------------------------------------------------------------------------
% These packages are required if you need to list code snippets.
\usepackage{listings}
% Nicely syntax highlighted m-code in LaTeX documents with stylefile mcode.sty
% http://www.mathworks.com/matlabcentral/fileexchange/8015-m-code-latex-package
\usepackage[numbered]{Tools/mcode}
%
% -----------------------------------------------------------------------------
% Re-define listings captions and titles based on language.
\newcaptionname{portuguese}{\lstlistingname}{Listagem} % Listings CAPTIONS
\newcaptionname{portuguese}{\lstlistlistingname}{Listagens} % LIST of LISTINGS
%
% -----------------------------------------------------------------------------
% PACKAGE csquotes
% -----------------------------------------------------------------------------
% Quotation helper package
\usepackage{csquotes}
%
% -----------------------------------------------------------------------------
% PACKAGE todonotes
% -----------------------------------------------------------------------------
% Create TODO Notes in text
% The notes can be made invisible by just using the 'disable' option:
\usepackage[textwidth=2cm, textsize=small]{todonotes}
%\usepackage[textwidth=2cm, textsize=small, disable]{todonotes}
\setlength{\marginparwidth}{2cm}
%
% -----------------------------------------------------------------------------
% PACKAGE changes
% -----------------------------------------------------------------------------
% Track changes in document (changes in pdf preview).
%% Use "final" option to make all tracking markups invisible.
%\usepackage[authormarkup=superscript,authormarkuptext=id,markup=underlined,ulem={ULforem,normalbf},final]{changes}
\usepackage[authormarkup=superscript,authormarkuptext=id,markup=underlined,ulem={ULforem,normalbf}]{changes}
% commands:
% \added[id=xx]{text}
% \deleted[id=xx]{text}
% \replaced[id=xx]{deleted text}{added text}
% -----------------------------------------------------------------------------
% PACKAGES xcolor, color
% -----------------------------------------------------------------------------
% These packages are required for list code snippets.
\usepackage{xcolor}
\usepackage{color}
% The following special color definitions are used in the IST Thesis
\definecolor{forestgreen}{RGB}{34,139,34}
\definecolor{orangered}{RGB}{239,134,64}
\definecolor{lightred}{rgb}{1,0.4,0.5}
\definecolor{orange}{rgb}{1,0.45,0.13}
\definecolor{darkblue}{rgb}{0.0,0.0,0.6}
\definecolor{lightblue}{rgb}{0.1,0.57,0.7}
\definecolor{gray}{rgb}{0.4,0.4,0.4}
\definecolor{lightgray}{rgb}{0.95, 0.95, 0.95}
\definecolor{darkgray}{rgb}{0.4, 0.4, 0.4}
\definecolor{editorGray}{rgb}{0.95, 0.95, 0.95}
\definecolor{editorOcher}{rgb}{1, 0.5, 0} % #FF7F00 -> rgb(239, 169, 0)
\definecolor{chaptergrey}{rgb}{0.6,0.6,0.6}
\definecolor{editorGreen}{rgb}{0, 0.5, 0} % #007C00 -> rgb(0, 124, 0)
\definecolor{olive}{rgb}{0.17,0.59,0.20}
\definecolor{brown}{rgb}{0.69,0.31,0.31}
\definecolor{purple}{rgb}{0.38,0.18,0.81}
%
% -----------------------------------------------------------------------------
% PACKAGE setspace:
% ----------------------------------------------------------------------------
% Provides support for setting the spacing between lines in a document.
% Package options include single spacing, one half spacing, and double spacing.
% Alternatively the spacing can be changed as required with:
% \singlespacing, \onehalfspacing, and \doublespacing commands
\usepackage{setspace}
%
% -----------------------------------------------------------------------------
% PACKAGE paralist
% -----------------------------------------------------------------------------
% This package provides the 'inparaenum' environment for inline lists
\usepackage{paralist}
% usage:
% \begin{inparaenum}[(a)]
% \item bla
% \item bla, bla
% \end{inparaenum}
% -----------------------------------------------------------------------------
% PACKAGE cite:
% -----------------------------------------------------------------------------
% The 'cite' package will result in citation numbers being automatically
% sorted and properly "ranged". i.e.,
% [1], [2], [5]--[7], [9]
\usepackage{cite}
%
% -----------------------------------------------------------------------------
% PACKAGE acronym:
% -----------------------------------------------------------------------------
% The package 'acronym' garantees that all acronyms definitions are
% given at the first usage.
% IMPORTANT: do not use acronyms in titles/captions; otherwise the definition
% will appear on the table of contents.
\usepackage[printonlyused]{acronym}
%
% -----------------------------------------------------------------------------
% PACKAGE hyperref
% -----------------------------------------------------------------------------
% Set links for references and citations in document
\usepackage{hyperref}
% pre-configuration of hyperref
\hypersetup{ colorlinks=false,
             citecolor=cyan,
             linkcolor=darkgray,
             urlcolor=teal,
             breaklinks=true,
             bookmarksnumbered=true,
             bookmarksopen=true,
             pdftitle=\@title, % THESIS TITLE
             pdfauthor=\@author,  % YOUR NAME
             pdfcreator=\@author,   % YOUR NAME
}
%
% -----------------------------------------------------------------------------
% PACKAGE url:
% -----------------------------------------------------------------------------
% Provides better support for handling and breaking URLs.
\usepackage{url}
%
% -----------------------------------------------------------------------------
% PACKAGE Cleveref:
% -----------------------------------------------------------------------------
% Clever Referencing of document parts
% Note: portuguese is supported through "brazilian" option
\usepackage[\IfLanguageName{english}{english}{brazilian}]{cleveref}
%
% -----------------------------------------------------------------------------
% PACKAGE enumitem:
% -----------------------------------------------------------------------------
%For enhanced enumeration of lists
%\usepackage{enumitem}
\usepackage[shortlabels]{enumitem}
\setlist[description]{leftmargin=\parindent,labelindent=\parindent,itemsep=1pt,parsep=0pt,topsep=0pt}
%
% #############################################################################
% GLOBAL FORMATTING OF THE THESIS DOCUMENT before using FANCY stuff
% Set paragraph counter to alphanumeric mode
\renewcommand{\theparagraph}{\Alph{paragraph}~--}
\hoffset 0in
\voffset 0in
\oddsidemargin 0 cm
\evensidemargin 0 cm
\marginparsep 0in
\topmargin -0.25cm
\textwidth 16 cm
\textheight 22.4 cm
\makeatletter
% package indentfirst says \let\@afterindentfalse\@afterindenttrue
% and we revert this modification, reinstating the original definitio
% of \@afterindentfalse
\def\@afterindentfalse{\let\if@afterindent\iffalse}
\makeatother
% -----------------------------------------------------------------------------
% PACKAGE fancyhdr:
% -----------------------------------------------------------------------------
% The fancyhdr macro package allows to customize page headers and footers.
\usepackage{fancyhdr}
\pagestyle{fancy}
\renewcommand{\chaptermark}[1]{\markboth{\thechapter.\ #1}{}}
\renewcommand{\sectionmark}[1]{\markright{\thesection\ #1}}
\fancyhead{}
\renewcommand{\headrulewidth}{0.0pt}
\renewcommand{\footrulewidth}{0.0pt}
\addtolength{\headheight}{2pt} % make space for the rule
\fancypagestyle{plain}{%
   \fancyhead{} % get rid of headers
   \renewcommand{\headrulewidth}{0pt} % and the line
   \renewcommand{\footrulewidth}{0pt}
}
\fancypagestyle{blank}{%
   \fancyhf{} % get rid of headers and footers
   \renewcommand{\headrulewidth}{0pt} % and the line
   \renewcommand{\footrulewidth}{0pt}
}
\fancypagestyle{abstract}{%
   \fancyhead{}
   \renewcommand{\headrulewidth}{0pt}
   \renewcommand{\footrulewidth}{0.0pt}
}
\fancypagestyle{document}{%
    \fancyhead{}
    \renewcommand{\headrulewidth}{0.5pt}
    \renewcommand{\footrulewidth}{0.5pt}
    \addtolength{\headheight}{2pt} % make space for the rule
}
\setcounter{secnumdepth} {5}
\setcounter{tocdepth} {5}
\renewcommand{\thesubsubsection}{\thesubsection.\Alph{subsubsection}}
%\renewcommand{\subfigtopskip}{0.3 cm}
%\renewcommand{\subfigbottomskip}{0.2 cm}
%\renewcommand{\subfigcapskip}{0.3 cm}
%\renewcommand{\subfigcapmargin}{0.2 cm}
%
% -----------------------------------------------------------------------------
% PACKAGE minitoc:
% -----------------------------------------------------------------------------
% Package 'minitoc' creates a mini-table of contents (a “minitoc”) at
% the beginning of each chapter of a document.
% This packages are required for the \fancychapter configuration
\usepackage{minitoc}
\setcounter{minitocdepth}{1}
\setlength{\mtcindent}{24pt}
\renewcommand{\mtcfont}{\small\rm}
\renewcommand{\mtcSfont}{\small\bf}
\renewcommand*{\kernafterminitoc}{\kern0.\baselineskip\kern0.ex}
\mtcselectlanguage{\languagename}
% Now prepare the MINITOC
\def\boxedverbatim{%
  \def\verbatim@processline{%
    {\setbox0=\hbox{\the\verbatim@line}%
    \hsize=\wd0 \the\verbatim@line\par}}%
  \@minipagetrue%%%DPC%%%
  \@tempswatrue%%%DPC%%%
  \setbox0=\vbox\bgroup\vspace*{0.2cm}\footnotesize\verbatim
}
\def\endboxedverbatim{%
  \endverbatim
  \unskip\setbox0=\lastbox %%%DPC%%%
  \hspace*{0.2cm}
  \vspace*{-0.2cm}
  \egroup
  \fbox{\box0}% <<<=== change here for centering,...
}
% Now prepare the CHAPTER Number
\newcommand*{\chapnumfont}{%
%   \usefont{T1}{\@defaultcnfont}{b}{n}\fontsize{100}{130}\selectfont%
  \usefont{T1}{pbk}{b}{n}
  \fontsize{150}{130}
  \selectfont
  \color{chaptergrey}
}
\makeatletter
\def\@makechapterhead#1{%
  \vspace*{50\p@}%
  {\parindent \z@ \raggedright \normalfont
    {\chapnumfont\ifnum \c@secnumdepth >\m@ne
%         \huge\bfseries \@chapapp\space \thechapter
        \raggedleft\bfseries \thechapter
        \par\nobreak
        \vskip 20\p@
    \fi}
    \interlinepenalty\@M
    {\raggedleft\Huge \bfseries #1\par\nobreak}
    \vskip 40\p@
  }}
\makeatother
% Now put it all together as a command \fancychapter
\newcommand{\fancychapter}[1]{\chapter{#1}\vfill\minitoc\pagebreak}
%
% #############################################################################
% ADDITIONAL COMMANDS AND CONFIGURATIONS
% #############################################################################
% This commmand allows to place horizontal lines with a custom width...
% replaces the standard hline command
\newcommand{\hlinew}[1]{%
  \noalign{\ifnum0=`}\fi\hrule \@height #1 \futurelet
   \reserved@a\@xhline}
%
% -----------------------------------------------------------------------------
% This command defines some marks... USEFUL FOR TABLES.
\def\Mark#1{\raisebox{0pt}[0pt][0pt]{\textsuperscript{\footnotesize\ensuremath{\ifcase#1\or *\or \dagger\or \ddagger\or%
    \mathsection\or \mathparagraph\or \|\or **\or \dagger\dagger%
    \or \ddagger\ddagger \else\textsuperscript{\expandafter\romannumeral#1}\fi}}}}
%
% -----------------------------------------------------------------------------
% The following configurations are used for LISTINGS of certain languages
\lstdefinestyle{XML} {
    language=XML,
    extendedchars=true,
    breaklines=true,
    breakatwhitespace=true,
    emph={},
    emphstyle=\color{red},
    basicstyle=\small,
    xleftmargin=17pt,
    columns=fullflexible,
    commentstyle=\color{gray}\upshape,
    morestring=[b][\color{brown}]",
    morecomment=[s]{<?}{?>},
    morecomment=[s][\color{forestgreen}]{<!--}{-->},
    keywordstyle=\color{orangered},
    stringstyle=\ttfamily\color{black},
    % stringstyle=\ttfamily\color{black}\normalfont,
    tagstyle=\color{blue},
    % tagstyle=\color{darkblue}\bf,
    morekeywords={asn,action,addrType,abilityNAT,audioSampleRate,audiChannels,,bandwidth,bitmapSize,bitRate,connection,codecs,concurrentLinks,dependency,duration,frameRate,from,height,ip,id,lang,mimeType,onlineTime,peerMode,port,priority,peerProtocol,property,release,to,tier,type,transactionID,url,uploadBWlevel,version,width},
    otherkeywords={attribute,xmlns,schemaLocation,PresentationType,availabilityStartTime,availabilityEndTime,minimumUpdatePeriod,minBufferTime,UpdateTime},
}
% ----------------------------------------------------------------------------
\lstdefinelanguage{Assembler}{
    morecomment=[l];,
    keywords={ADD,ADDC,SUB,SUBB,CMP,MUL,DIV,MOD,NEG,AND,OR,NOT,XOR,TEST,BIT,SET,EI,EI0,EI1,EI2,EI3,SETC,EDMA,CLR,DI,DI0,DI1,DI2,DI3,CLRC,SHR,SHL,SHRA,SHLA,ROR,ROL,RORC,ROLC,MOV,MOVB,MOVBS,MOVP,MOVL,MOVH,SWAP,PUSH,POP,JZ,JNZ,JN,JNN,JP,JNP,JC,JNC,JV,JNV,JEQ,JNE,JLT,JLE,JGT,JGE,JA,JAE,JB,JBE,JMP,CALL,CALLF,RET,RETF,SWE,RFE,NOP},
    morekeywords={EQU,TABLE,WORD,STRING,PLACE},
}
% ----------------------------------------------------------------------------
\lstdefinestyle{coloredASM}{
    language=Assembler,
    extendedchars=false,
    breaklines=true,
    tabsize=2,
    numberstyle=\tiny,
    numbers=left,
    breakatwhitespace=true,
    emph={},
    emphstyle=\color{red},
    fontadjust=true,
    basicstyle=\small\ttfamily,
    % basicstyle=\footnotesize\ttfamily,
    columns=fixed,
    xleftmargin=17pt,
    framexleftmargin=17pt,
    framexrightmargin=5pt,
    framexbottommargin=4pt,
    commentstyle=\color{forestgreen}\upshape,
    morestring=[b][\color{brown}]",
    keywordstyle=\color{darkblue},
    stringstyle=\ttfamily\color{black},
    literate={á}{{\'a}}1 {ã}{{\~a}}1 {â}{{\^a}}1 {é}{{\'e}}1 {É}{{\'E}}1 {ê}{{\^e}}1 {õ}{{\~o}}1 {ó}{{\'o}}1 {í}{{\'i}}1 {ç}{{\c{c}}}1 {Ç}{{\c{C}}}1,
}
% ----------------------------------------------------------------------------
\lstdefinelanguage{CSS}{
    sensitive=true,
    morecomment=[l]{//},
    morecomment=[s]{/*}{*/},
    morestring=[b]',
    morestring=[b]",
    alsoletter={:},
    alsodigit={-},
    keywords={color,background-image:,margin,padding,font,weight,display,position,top,left,right,bottom,list,style,border,size,white,space,min,width, transition:, transform:, transition-property, transition-duration, transition-timing-function}
}
% ----------------------------------------------------------------------------
% JavaScript
\lstdefinelanguage{JavaScript}{
    morecomment=[s]{/*}{*/},
    morecomment=[l]//,
    morestring=[b]",
    morestring=[b]',
    morekeywords={typeof, new, true, false, catch, function, return, null, catch, switch, var, if, in, while, do, else, case, break}
}
% ----------------------------------------------------------------------------
\lstdefinelanguage{HTML5}{
    language=html,
    sensitive=true,
    alsoletter={<>=-},
    morecomment=[s]{<!-}{-->},
    tag=[s],
    otherkeywords={
    % General
    >,
    % Standard tags
    <!DOCTYPE,
    </html, <html, <head, <title, </title, <style, </style, <link, </head, <meta, />,
    % body
    </body, <body,
    % Divs
    </div, <div, </div>,
    % Paragraphs
    </p, <p, </p>,
    % scripts
    </script, <script,
    % More tags...
    <canvas, /canvas>, <svg, <rect, <animateTransform, </rect>, </svg>, <video, <source, <iframe, </iframe>, </video>, <image, </image>, <header, </header, <article, </article},
    ndkeywords={
    % General
    =,
    % HTML attributes
    charset=, src=, id=, width=, height=, style=, type=, rel=, href=,
    % SVG attributes
    fill=, attributeName=, begin=, dur=, from=, to=, poster=, controls=, x=, y=, repeatCount=, xlink:href=,
    % properties
    margin:, padding:, background-image:, border:, top:, left:, position:, width:, height:, margin-top:, margin-bottom:, font-size:, line-height:,
    % CSS3 properties
    transform:, -moz-transform:, -webkit-transform:,
    animation:, -webkit-animation:,
    transition:,  transition-duration:, transition-property:, transition-timing-function:,
    }
}
% ----------------------------------------------------------------------------
\lstdefinestyle{htmlcssjs} {%
    % General design
    backgroundcolor=\color{editorGray},
        fontadjust=true,
    basicstyle=\small\ttfamily,
    frame=b,
    % line-numbers
    xleftmargin={0.75cm},
    numbers=left,
    stepnumber=1,
    firstnumber=1,
    numberfirstline=true,
    % Code design
    identifierstyle=\color{black},
    keywordstyle=\color{blue}\bfseries,
    ndkeywordstyle=\color{editorGreen}\bfseries,
    stringstyle=\color{editorOcher}\ttfamily,
    commentstyle=\color{brown}\ttfamily,
    % Code
    language=HTML5,
    alsolanguage=JavaScript,
    alsodigit={.:;},
    tabsize=2,
    showtabs=false,
    showspaces=false,
    showstringspaces=false,
    extendedchars=true,
    breaklines=true,
    % German umlauts
    literate=%
    {Ö}{{\"O}}1
    {Ä}{{\"A}}1
    {Ü}{{\"U}}1
    {ß}{{\ss}}1
    {ü}{{\"u}}1
    {ä}{{\"a}}1
    {ö}{{\"o}}1
}
% ----------------------------------------------------------------------------
\lstdefinestyle{py} {%
    language=python,
    literate=%
    *{0}{{{\color{lightred}0}}}1
    {1}{{{\color{lightred}1}}}1
    {2}{{{\color{lightred}2}}}1
    {3}{{{\color{lightred}3}}}1
    {4}{{{\color{lightred}4}}}1
    {5}{{{\color{lightred}5}}}1
    {6}{{{\color{lightred}6}}}1
    {7}{{{\color{lightred}7}}}1
    {8}{{{\color{lightred}8}}}1
    {9}{{{\color{lightred}9}}}1,
    basicstyle=\small\ttfamily,
    numbers=left,
    % numberstyle=\tiny,
    % stepnumber=2,
    numbersep=5pt,
    tabsize=4,
    extendedchars=true,
    breaklines=true,
    keywordstyle=\color{blue}\bfseries,
    frame=b,
    commentstyle=\color{brown}\itshape,
    stringstyle=\color{editorOcher}\ttfamily,
    showspaces=false,
    showtabs=false,
    xleftmargin=17pt,
    framexleftmargin=17pt,
    framexrightmargin=5pt,
    framexbottommargin=4pt,
    backgroundcolor=\color{lightgray},
    showstringspaces=false,
}

%
% #############################################################################
% #############################################################################
\begin{document}
%
% Add PDF bookmark
\pdfbookmark[0]{Titlepage}{Title}
% #############################################################################
% DEFINE THE Front Cover Page of Thesis-MSc
% !TEX root = ./main.tex
% #############################################################################
% Thesis-MSc
% Version 2.0, August 2018
% BY: Rui Santos Cruz, rui.s.cruz@tecnico.ulisboa.pt
% #############################################################################
%
% REQUIRED LOGO:
% The university logo image: arguments correspond to {left}{top} position. 
% IST rules determine the position to be be 2cm from top, left page edge
\univlogo{2cm}{2cm}{./Images/IST_A_RGB_POS}
% OPTIONAL IMAGE:
% The thesis image: arguments are the start position in the page.
% You can change the image for your thesis, replacing the image name:
%\thesislogo{2.5cm}{6cm}{./Images/thesis_logo}
%\thesislogo{2.5cm}{6cm}{./Images/tecnico-lisboa}
% \thesislogo{2.5cm}{6cm}{./Images/capa_2}
%\thesislogo{2.5cm}{6cm}{./Images/tecnico-lisboa}
%\thesislogo{2.5cm}{6cm}{./Images/capa_3}
%
% -----------------------------------------------------------------------------
% REQUIRED: Thesis TITLE
\title{REPLACE\_WITH\_THESIS\_NAME}
% OPTIONAL: Thesis SUBTITLE
%\subtitle{This is the Thesis Subtitle if Necessary}
%
% -----------------------------------------------------------------------------
% REQUIRED: Author
% Author full Name
\author{REPLACE\_WITH\_YOUR\_NAME}
%
% -----------------------------------------------------------------------------
% The official name of the course/degree. Please chose portuguese or english
% un-comment the line corresponding to your degree.
% You can add a degree name using this construct
%
\degree{REPLACE\_WITH\_DEGREE\_NAME}
%
% -----------------------------------------------------------------------------
% REQUIRED: The SUPERVISOR(s) - maximum of two
\supervisor{REPLACE\_WITH\_MAIN\_SUPERVISOR\_NAME}
%\supervisor{Prof. Luis Russo}
% If no co-Supervisor comment the next line
\othersupervisor{REPLACE\_WITH\_CO\_SUPERVISOR\_NAME}
%
% -----------------------------------------------------------------------------
% REQUIRED: Date of examination
% Insert the Date of the Thesis discussion (format is MONTH and YEAR)
\date{October 2019}
%
% -----------------------------------------------------------------------------
% The following command define the author colors for Tracking Changes in doc
\definechangesauthor[color=forestgreen]{AA}
\definechangesauthor[color=blue]{BB}
\definechangesauthor[color=red]{CC}

% -----------------------------------------------------------------------------
% Place 'false' when delivering the draft version of the thesis.
% The committee members should not be printed for the draft version. 
% Place 'true' after the Examination Committee has accepted the thesis as final
%\finalthesis{true}
\finalthesis{false}
%
% -----------------------------------------------------------------------------
% The members of the Examination Committee
%\chairperson{Prof. Name of the Chairperson}
%\vogalone{Prof. Name of First Committee Member}
%\vogaltwo{Dr. Name of Second Committee Member}
%\vogalthree{Eng. Name of Third Committee Member}
%
% -----------------------------------------------------------------------------
% Please DO NOT MODIFY the following lines.
% print the titlepage
\maketitle
\clearpage
\thispagestyle{empty}
% If Printing on DOUBLE SIDED pages, the second page should be white.

\cleardoublepage
%
% -----------------------------------------------------------------------------
% PAGE NUMBERING FOR INDEXING MATTER in ROMAN
\setcounter{page}{1} \pagenumbering{roman}
\baselineskip 18pt % line spacing: -12pt for single spacing
                   %               -18pt for 1 1/2 spacing
                   %               -24pt for double spacing
% -----------------------------------------------------------------------------
% -----------------------------------------------------------------------------
% THE ABSTRACT
\pdfbookmark[0]{Abstract}{Abstract}
\begin{abstract}
   % #############################################################################
% Abstract Text
% !TEX root = ../main.tex
% #############################################################################
% use \noindent in first paragraph
\noindent




\end{abstract}
\clearpage
%\begin{keywords}
%   \input{./Chapters/Thesis-MSc-Abstract-EN-KeyWords.tex}
%\end{keywords}
%\clearpage
%\thispagestyle{empty}
%% If Printing on DOUBLE SIDED pages, the second page should be white.
%% Otherwise, comment the following command:
%\cleardoublepage
%
% -----------------------------------------------------------------------------
% O RESUMO
%\pdfbookmark[0]{Resumo}{Resumo}
%\begin{resumo}
%   \input{./Chapters/Thesis-MSc-Abstract-PT-Resumo.tex}
%\end{resumo}
%\begin{palavraschave}
%   \input{./Chapters/Thesis-MSc-Abstract-PT-PalavrasChave.tex}
%\end{palavraschave}
%\clearpage
%\thispagestyle{empty}
%% If Printing on DOUBLE SIDED pages, the second page should be white.
%% Otherwise, comment the following command:
%\cleardoublepage
%
% -----------------------------------------------------------------------------
% This is required for the Fancy Chapters with minitoc
\dominitoc
%\dominilof
%\dominilot


% -----------------------------------------------------------------------------
%                                   Lists of Contents
\renewcommand{\baselinestretch}{1}
\pdfbookmark[0]{Contents}{toc}
\tableofcontents
%\contentsline{chapter}{References}{\pageref{bib}}
% If Printing on DOUBLE SIDED pages, the second page should be white.
% Otherwise, comment the following command:
\cleardoublepage
% reposition baseline
\renewcommand{\baselinestretch}{1.5}
% -----------------------------------------------------------------------------







% List of Figures
%\pdfbookmark[1]{List of Figures}{lof}
%\listoffigures
%\cleardoublepage
% -----------------------------------------------------------------------------
%\begingroup
%    \let\clearpage\relax
%    \let\cleardoublepage\relax
%    \let\cleardoublepage\relax
% List of Tables
%\pdfbookmark[1]{List of Tables}{lot}
%\listoftables
% If Printing on DOUBLE SIDED pages, the second page should be white.
% Otherwise, comment the following command:
%\let\cleardoublepage\relax
%\cleardoublepage
%\clearpage
% -----------------------------------------------------------------------------
% List of Algorithms
% If not used, comments the lines!
% Requires packages algorithmic, algorithm
%\pdfbookmark[1]{List of Algorithms}{loa}
%\listofalgorithms
% If Printing on DOUBLE SIDED pages, the second page should be white.
%\endgroup
% Otherwise, comment the following command:
%\cleardoublepage
% -----------------------------------------------------------------------------
% Listings
% If not used, comments the lines!
% Requires packages listings
%\pdfbookmark[1]{Listings}{lol}
%\lstlistoflistings
%\cleardoublepage
% -----------------------------------------------------------------------------
% % List of acronyms
\pdfbookmark[1]{Acronyms}{loac}
\chapter*{\tlangAcronyms}
% #############################################################################
% This is the ACRONYMS Definition
% !TEX root = ../main.tex


% #############################################################################


\begin{acronym}[H.264/SVC]

    \acro{ist}[IST]{Instituto Superior T\'ecnico}
    \acro{tsb}[TSB]{Técnico Solar Boat}
    \acro{mppt}[MPPT]{Maximum Power Point Tracker}
    \acro{mpp}[MPP]{Maximum Power Point}
    \acro{pcb}[PCB]{Printed Circuit Board}
    \acro{can}[CAN]{Controller Area Network}
    \acro{usb}[USB]{Universal Serial Bus}
    \acro{gui}[GUI]{Graphical User Interface}

\end{acronym}


% #############################################################################
%           How to define acronyms
% \acro{<label>}[<short>]{<long>}
% #############################################################################
%           How to use acronyms
% \ac{<label>}
%   Prints the full acronym (first use: full phrase + abbreviation; later uses: abbreviation only).
%   Example: \ac{IST} → "Instituto Superior Técnico (IST)" (first use), "IST" (subsequent uses).

% \acs{<label>}
%   Prints the short form (just the abbreviation).
%   Example: \acs{IST} → "IST"

% \acl{<label>}
%   Prints the long form (just the full phrase).
%   Example: \acl{IST} → "Instituto Superior Técnico"

% \acf{<label>}
%   Prints the full form (full phrase + abbreviation) every time.
%   Example: \acf{IST} → "Instituto Superior Técnico (IST)"

% \acp{<label>}
%   Prints the plural form of the acronym.
%   Example: \acp{IST} → "ISTs"
% #############################################################################
% If Printing on DOUBLE SIDED pages, the second page should be white.
% Otherwise, comment the following command:
\cleardoublepage
% -----------------------------------------------------------------------------
% PAGE NUMBERING FOR DOCUMENT MATTER in ARABIC
% Pages number is starting with arabic style. Until here were on roman mode
\setcounter{page}{1} \pagenumbering{arabic}
\baselineskip 18pt



% -----------------------------------------------------------------------------
%                                   CHAPTERS



%Chapter 1
\acresetall
\fancychapter{Introduction}
\label{chap:introduction}
% In total 3-4 pages

% Motivation [2]
\section{Motivation}




    % Context [1.5]
    % Overall view of the problem [0.5]

% Background 
\input{Chapters/Introduction/background}
    % Solar Panels [0.5]
    % MPPTs [0.5]
    % TSB Project [0.5]

% Objectives [1]
\section{Objectives}





% Document Structure [1]
\section{Outline}

\textcolor{red}{Explain how the work is organized by chapters.}





\cleardoublepage

\cleardoublepage
% -----------------------------------------------------------------------------


%Chapter 2
\acresetall
\fancychapter{State-of-the-Art}
\label{chap:SoA}

%In total about 15 pages
% Brief introduction to the SoA
\textcolor{red}{Intro if needed}




% ...state of the art body...

% Conclusions on the SoA
\section{Summary}




\cleardoublepage

\cleardoublepage
% -----------------------------------------------------------------------------

%Chapter 3
\acresetall
\fancychapter{Solucion Proposal}
\label{chap:Solucion_Proposal}



\cleardoublepage

\cleardoublepage
% -----------------------------------------------------------------------------

%Chapter 4
\acresetall
\fancychapter{Preliminary Work}
\label{chap:conclusion}







\cleardoublepage

\cleardoublepage
% -----------------------------------------------------------------------------

%Chapter 4
\acresetall
\fancychapter{Planning and Scheduling}
\label{chap:PandS}


\textcolor{red}{Fazer um planeamento com um gantt chart e explicar as decisoes}




\cleardoublepage
\cleardoublepage
% -----------------------------------------------------------------------------

%Chapter 5
%\acresetall
%\input{Chapters/}
%\cleardoublepage
% -----------------------------------------------------------------------------















% BIBLIOGRAPHY
% Add the Bibliography to the PDF table of contents (not the document table of contents)
\pdfbookmark[0]{Bibliography}{bib}
% The bibliography style sheet
% Chose your preferences on the format of the entries and the Labels:
% IEEEtran: Used in general (recommended for IST Thesis)
%           Entries are labelled and sorted by appearance in the document
%           Labels are Numeric inside square brackets
\bibliographystyle{IEEEtran}
%
% Apalike:  Entries formatted alphabetically, last name first, with identation
%           Labels with Autor's Name and Year inside square brackets
%\bibliographystyle{apalike}
%
% Alpha:    Entries formatted with Autor's Name and Year, hanging identation
%           Labels with Autor's abbr. Names and Year inside square brackets
%\bibliographystyle{alpha}
%
% Acm:     Entries formatted with Autor's Name (small Caps), hanging identation
%          Labels are Numeric inside square brackets
%\bibliographystyle{acm}
% The following command resets the 'emphasis' style for bibliography entries
\normalem
% Name of your BiBTeX file
\bibliography{Bibliography/Bibliography} % Put here your own filename
%
% The following command modifies the 'emphasis' style for bibliography entries
\ULforem
% If Printing on DOUBLE SIDED pages, the second page should be white.
% Otherwise, comment the following command:
%\cleardoublepage
%
% -----------------------------------------------------------------------------
% HERE GO THE APPENDIXES IF REQUIRED
% If not required just comment the blocks
\appendix
%% First Appendix

\pdfbookmark[1]{Appendix A}{appendix}
% #############################################################################
% This is Appendix A
% !TEX root = ../main.tex
% #############################################################################
\chapter{Appendix Name}
\label{chapter:appendixA}



%% If Printing on DOUBLE SIDED pages, the second page should be white.
%% Otherwise, comment the following command:
%\cleardoublepage
%% Second Appendix


%\pdfbookmark[1]{Appendix B}{appendix}
%\input{./Chapters/Thesis-MSc-AppendixB.tex}

%% If Printing on DOUBLE SIDED pages, the second page should be white.
%% Otherwise, comment the following command:
%\cleardoublepage

% -----------------------------------------------------------------------------
% And this is THE END of the IST Thesis Document
\end{document}
