% #############################################################################
% Abstract Text
% !TEX root = ../main.tex
% #############################################################################
% use \noindent in first paragraph
\noindent

Solar-powered vehicles rely on solar energy to charge their batteries and maximize their range, particularly solar-powered boats need an effective energy conversion in rapidly changing environmental conditions. In this context, The \ac{mppt} charger plays a critical role in energy conversion, by ensuring that the solar panel operates on their \ac{mpp} while charging a battery. This work focus on the study, design, simulation and test of a fast response \ac{mppt} charger for \ac{tsb}.

This document contains the state-of-the-art review covering photovoltaic principles, \ac{mppt} algorithms, DC–DC converter topologies, charging techniques, processing units, and commercial solutions. Special attention is given to the algorithms suitable for dynamic conditions, and DC–DC topologies. 

Based on the conduct study and the project requirements, a solution was proposed capable of effectively convert energy, operating the solar panel on their \ac{mpp}, and without compromising the product cost. Preliminary simulations were conduct to validate the software tools chosen and gain familiarity with the given tools. The ultimate objective of this project is to develop a reliable, efficient, and customizable \ac{mppt} solution that improves energy extraction, enhances system monitoring through CAN communication, and contributes to the overall performance and competitiveness of the \ac{tsb}.

\vspace{8cm}