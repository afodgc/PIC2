% #############################################################################
% This is the ACRONYMS Definition
% !TEX root = ../main.tex


% #############################################################################


\begin{acronym}[H.264/mppt]

    \acro{ist}[IST]{Instituto Superior T\'ecnico}
    \acro{tsb}[TSB]{Técnico Solar Boat}
    \acro{mppt}[MPPT]{Maximum Power Point Tracker}
    \acro{mpp}[MPP]{Maximum Power Point}
    \acro{pcb}[PCB]{Printed Circuit Board}
    \acro{can}[CAN]{Controller Area Network}
    \acro{usb}[USB]{Universal Serial Bus}
    \acro{gui}[GUI]{Graphical User Interface}

\end{acronym}


% #############################################################################
%           How to define acronyms
% \acro{<label>}[<short>]{<long>}
% #############################################################################
%           How to use acronyms
% \ac{<label>}
%   Prints the full acronym (first use: full phrase + abbreviation; later uses: abbreviation only).
%   Example: \ac{IST} → "Instituto Superior Técnico (IST)" (first use), "IST" (subsequent uses).

% \acs{<label>}
%   Prints the short form (just the abbreviation).
%   Example: \acs{IST} → "IST"

% \acl{<label>}
%   Prints the long form (just the full phrase).
%   Example: \acl{IST} → "Instituto Superior Técnico"

% \acf{<label>}
%   Prints the full form (full phrase + abbreviation) every time.
%   Example: \acf{IST} → "Instituto Superior Técnico (IST)"

% \acp{<label>}
%   Prints the plural form of the acronym.
%   Example: \acp{IST} → "ISTs"
% #############################################################################